% Options for packages loaded elsewhere
\PassOptionsToPackage{unicode}{hyperref}
\PassOptionsToPackage{hyphens}{url}
%
\documentclass[
]{article}
\usepackage{lmodern}
\usepackage{amssymb,amsmath}
\usepackage{ifxetex,ifluatex}
\ifnum 0\ifxetex 1\fi\ifluatex 1\fi=0 % if pdftex
  \usepackage[T1]{fontenc}
  \usepackage[utf8]{inputenc}
  \usepackage{textcomp} % provide euro and other symbols
\else % if luatex or xetex
  \usepackage{unicode-math}
  \defaultfontfeatures{Scale=MatchLowercase}
  \defaultfontfeatures[\rmfamily]{Ligatures=TeX,Scale=1}
\fi
% Use upquote if available, for straight quotes in verbatim environments
\IfFileExists{upquote.sty}{\usepackage{upquote}}{}
\IfFileExists{microtype.sty}{% use microtype if available
  \usepackage[]{microtype}
  \UseMicrotypeSet[protrusion]{basicmath} % disable protrusion for tt fonts
}{}
\makeatletter
\@ifundefined{KOMAClassName}{% if non-KOMA class
  \IfFileExists{parskip.sty}{%
    \usepackage{parskip}
  }{% else
    \setlength{\parindent}{0pt}
    \setlength{\parskip}{6pt plus 2pt minus 1pt}}
}{% if KOMA class
  \KOMAoptions{parskip=half}}
\makeatother
\usepackage{xcolor}
\IfFileExists{xurl.sty}{\usepackage{xurl}}{} % add URL line breaks if available
\IfFileExists{bookmark.sty}{\usepackage{bookmark}}{\usepackage{hyperref}}
\hypersetup{
  pdftitle={Math bootcamp Assignment},
  hidelinks,
  pdfcreator={LaTeX via pandoc}}
\urlstyle{same} % disable monospaced font for URLs
\usepackage[margin=1in]{geometry}
\usepackage{color}
\usepackage{fancyvrb}
\newcommand{\VerbBar}{|}
\newcommand{\VERB}{\Verb[commandchars=\\\{\}]}
\DefineVerbatimEnvironment{Highlighting}{Verbatim}{commandchars=\\\{\}}
% Add ',fontsize=\small' for more characters per line
\usepackage{framed}
\definecolor{shadecolor}{RGB}{248,248,248}
\newenvironment{Shaded}{\begin{snugshade}}{\end{snugshade}}
\newcommand{\AlertTok}[1]{\textcolor[rgb]{0.94,0.16,0.16}{#1}}
\newcommand{\AnnotationTok}[1]{\textcolor[rgb]{0.56,0.35,0.01}{\textbf{\textit{#1}}}}
\newcommand{\AttributeTok}[1]{\textcolor[rgb]{0.77,0.63,0.00}{#1}}
\newcommand{\BaseNTok}[1]{\textcolor[rgb]{0.00,0.00,0.81}{#1}}
\newcommand{\BuiltInTok}[1]{#1}
\newcommand{\CharTok}[1]{\textcolor[rgb]{0.31,0.60,0.02}{#1}}
\newcommand{\CommentTok}[1]{\textcolor[rgb]{0.56,0.35,0.01}{\textit{#1}}}
\newcommand{\CommentVarTok}[1]{\textcolor[rgb]{0.56,0.35,0.01}{\textbf{\textit{#1}}}}
\newcommand{\ConstantTok}[1]{\textcolor[rgb]{0.00,0.00,0.00}{#1}}
\newcommand{\ControlFlowTok}[1]{\textcolor[rgb]{0.13,0.29,0.53}{\textbf{#1}}}
\newcommand{\DataTypeTok}[1]{\textcolor[rgb]{0.13,0.29,0.53}{#1}}
\newcommand{\DecValTok}[1]{\textcolor[rgb]{0.00,0.00,0.81}{#1}}
\newcommand{\DocumentationTok}[1]{\textcolor[rgb]{0.56,0.35,0.01}{\textbf{\textit{#1}}}}
\newcommand{\ErrorTok}[1]{\textcolor[rgb]{0.64,0.00,0.00}{\textbf{#1}}}
\newcommand{\ExtensionTok}[1]{#1}
\newcommand{\FloatTok}[1]{\textcolor[rgb]{0.00,0.00,0.81}{#1}}
\newcommand{\FunctionTok}[1]{\textcolor[rgb]{0.00,0.00,0.00}{#1}}
\newcommand{\ImportTok}[1]{#1}
\newcommand{\InformationTok}[1]{\textcolor[rgb]{0.56,0.35,0.01}{\textbf{\textit{#1}}}}
\newcommand{\KeywordTok}[1]{\textcolor[rgb]{0.13,0.29,0.53}{\textbf{#1}}}
\newcommand{\NormalTok}[1]{#1}
\newcommand{\OperatorTok}[1]{\textcolor[rgb]{0.81,0.36,0.00}{\textbf{#1}}}
\newcommand{\OtherTok}[1]{\textcolor[rgb]{0.56,0.35,0.01}{#1}}
\newcommand{\PreprocessorTok}[1]{\textcolor[rgb]{0.56,0.35,0.01}{\textit{#1}}}
\newcommand{\RegionMarkerTok}[1]{#1}
\newcommand{\SpecialCharTok}[1]{\textcolor[rgb]{0.00,0.00,0.00}{#1}}
\newcommand{\SpecialStringTok}[1]{\textcolor[rgb]{0.31,0.60,0.02}{#1}}
\newcommand{\StringTok}[1]{\textcolor[rgb]{0.31,0.60,0.02}{#1}}
\newcommand{\VariableTok}[1]{\textcolor[rgb]{0.00,0.00,0.00}{#1}}
\newcommand{\VerbatimStringTok}[1]{\textcolor[rgb]{0.31,0.60,0.02}{#1}}
\newcommand{\WarningTok}[1]{\textcolor[rgb]{0.56,0.35,0.01}{\textbf{\textit{#1}}}}
\usepackage{graphicx,grffile}
\makeatletter
\def\maxwidth{\ifdim\Gin@nat@width>\linewidth\linewidth\else\Gin@nat@width\fi}
\def\maxheight{\ifdim\Gin@nat@height>\textheight\textheight\else\Gin@nat@height\fi}
\makeatother
% Scale images if necessary, so that they will not overflow the page
% margins by default, and it is still possible to overwrite the defaults
% using explicit options in \includegraphics[width, height, ...]{}
\setkeys{Gin}{width=\maxwidth,height=\maxheight,keepaspectratio}
% Set default figure placement to htbp
\makeatletter
\def\fps@figure{htbp}
\makeatother
\setlength{\emergencystretch}{3em} % prevent overfull lines
\providecommand{\tightlist}{%
  \setlength{\itemsep}{0pt}\setlength{\parskip}{0pt}}
\setcounter{secnumdepth}{-\maxdimen} % remove section numbering

\title{Math bootcamp Assignment}
\author{}
\date{\vspace{-2.5em}}

\begin{document}
\maketitle

\hypertarget{q1.-explain-the-role-of-matrix-algebra-in-data-science.}{%
\subsection{Q1. Explain the role of matrix algebra in data
science.}\label{q1.-explain-the-role-of-matrix-algebra-in-data-science.}}

\begin{Shaded}
\begin{Highlighting}[]
\KeywordTok{summary}\NormalTok{(cars)}
\end{Highlighting}
\end{Shaded}

\begin{verbatim}
##      speed           dist       
##  Min.   : 4.0   Min.   :  2.00  
##  1st Qu.:12.0   1st Qu.: 26.00  
##  Median :15.0   Median : 36.00  
##  Mean   :15.4   Mean   : 42.98  
##  3rd Qu.:19.0   3rd Qu.: 56.00  
##  Max.   :25.0   Max.   :120.00
\end{verbatim}

\hypertarget{q2.-consider-two-matrices-a-and-b.-dimensions-for-both-the-matrices-is-nxn.-when-will-the-following-identity-be-true}{%
\subsection{Q2. Consider two matrices A and B. Dimensions for both the
matrices is nxn. When will the following identity be
true:}\label{q2.-consider-two-matrices-a-and-b.-dimensions-for-both-the-matrices-is-nxn.-when-will-the-following-identity-be-true}}

\((A+B){^2}=A^2+B^2+2AB\)

\hypertarget{q3.}{%
\subsection{Q3.}\label{q3.}}

\[
A= \left(\begin{array}{cc} 
1 & 2 & 3\\
5 & 6 & 7
\end{array}\right)
B=
\left(\begin{array}{cc} 
-1 & -3\\ 
-5 & 6\\
7 & 8
\end{array}\right)
\] a. Find \(AB\) Show all steps b. Find \(BA\). Show all steps

\hypertarget{q4.-choose-one-or-more-names-normal-idempotent-nilpotent-or-unipotent-for-the-following-matrix}{%
\subsection{Q4. Choose one or more names (normal, idempotent, nilpotent
or unipotent) for the following
matrix}\label{q4.-choose-one-or-more-names-normal-idempotent-nilpotent-or-unipotent-for-the-following-matrix}}

\[ a. \left(\begin{array}{cc} 
1 & x\\
0 & -1
\end{array}\right) \]

\hypertarget{q5.-for-the-following-matrix-calculate-the-inverse.-show-all-the-steps.}{%
\subsection{Q5. For the following matrix, calculate the inverse. Show
all the
steps.}\label{q5.-for-the-following-matrix-calculate-the-inverse.-show-all-the-steps.}}

\[ a. \left(\begin{array}{cc} 
1 & -3\\
5 & -7
\end{array}\right) \]

\hypertarget{q6.-find-the-rank-of-the-matrix-eigen-values-and-vectors-of-the-following-matrix.-show-all-the-steps.}{%
\subsection{Q6. Find the rank of the matrix, eigen values and vectors of
the following matrix. Show all the
steps.}\label{q6.-find-the-rank-of-the-matrix-eigen-values-and-vectors-of-the-following-matrix.-show-all-the-steps.}}

\[
a. \left(\begin{array}{cc} 
1 & 2 & 4\\
3 & 8 & 14\\
2 & 6 & 13
\end{array}\right)
\] \#\# Q7. Prove

\((AB)^{-1} = B^{-1}A^{-1}\)

\hypertarget{q8.-read-this-article-on-tensor-flow.-httpsmachinelearningmastery.comintroduction-to-tensors-for-machine-learning.-write-a-note-on-what-are-tensors-how-are-they-different-from-vectors-and-matrices-how-are-they-helpful-discuss-one-application-of-tensors-where-matrices-and-vectors-cant-be-used.}{%
\subsection{\texorpdfstring{Q8. Read this article on tensor flow.
(\url{https://machinelearningmastery.com/introduction-to-tensors-for-machine-learning/}).
Write a note on what are tensors? How are they different from vectors
and matrices? How are they helpful? Discuss one application of tensors
where matrices and vectors can't be
used.}{Q8. Read this article on tensor flow. (https://machinelearningmastery.com/introduction-to-tensors-for-machine-learning/). Write a note on what are tensors? How are they different from vectors and matrices? How are they helpful? Discuss one application of tensors where matrices and vectors can't be used.}}\label{q8.-read-this-article-on-tensor-flow.-httpsmachinelearningmastery.comintroduction-to-tensors-for-machine-learning.-write-a-note-on-what-are-tensors-how-are-they-different-from-vectors-and-matrices-how-are-they-helpful-discuss-one-application-of-tensors-where-matrices-and-vectors-cant-be-used.}}

\hypertarget{q9.-using-lu-decomposition-decompose-the-following-matrix-in-to-an-upper-and-a-lower-triangular-matrix.-show-all-the-steps}{%
\subsection{Q9. Using LU decomposition, decompose the following matrix
in to an upper and a lower triangular matrix. Show all the
steps}\label{q9.-using-lu-decomposition-decompose-the-following-matrix-in-to-an-upper-and-a-lower-triangular-matrix.-show-all-the-steps}}

\[
a. \left(\begin{array}{cc} 
1 & 2 & 4\\
3 & 8 & 14\\
2 & 6 & 13
\end{array}\right)
\]

\end{document}
